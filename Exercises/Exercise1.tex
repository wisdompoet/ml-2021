
\documentclass{article}

\usepackage{color}
\usepackage{epsfig}
\usepackage{amssymb}
\usepackage{listings}
\usepackage{hyperref}

%\usetheme{Berlin}
%\usecolortheme{seahorse}
%\usefonttheme{professionalfonts}

\begin{document}

  \section*{Programming Exercise 1}

  {\bf Due on 26 October}\\


In this task we will focus on the training set only.
Consider the following repository of datasets:\\

\noindent
\url{http://www.csie.ntu.edu.tw/~cjlin/libsvmtools/datasets/}\\

\noindent

\section{Regression Task}

Choose a regression dataset and apply linear regression on a random subset of the training set of increasing size. You should select training sets that include more and more data points.
\begin{enumerate}
\item Plot the approximation error (square loss) on the training set as a function of the number of samples $N$ (i.e.~data points in the training set). % change to m !!!
\item Plot the cpu-time as a function of $N$.
\item Explain in detail the behaviour of both curves.
\item Explore how the learned weights change as a function of $N$.
Can you find an interpretation for the learned weights?
\end{enumerate}

\section{Classification Task}

Choose a classification dataset and apply logistic regression.
Repeat the previous steps using as error the mean accuracy.


\section{Hints}
\begin{itemize}
\item Feel free to use your favourite software. We recommend Python.
\item You can obtain smoother curves by averaging over several permutations of the dataset.
Ideally you could also plot the variance, not only the mean.
\item In this exercise we are only evaluating in-sample error, \textbf{do not use regularization!}
\item Check your results are in agreement with the theory.
\item Useful links:
\\
\noindent
\url{https://scikit-learn.org/stable/user_guide.html}\\
\url{https://colab.research.google.com/}\\
\end{itemize}


\end{document}
